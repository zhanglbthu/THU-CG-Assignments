\documentclass[12pt,a4paper]{article}
\usepackage[margin=1in]{geometry} % 设置页面边距
\usepackage{ctex}
\usepackage{graphicx}
\usepackage{float}
\usepackage{amsmath}
\usepackage{amssymb} % 数学符号
\usepackage{booktabs} % 绘制漂亮的表格
\usepackage{array} % 数组和表格
\usepackage{multirow} % 多行合并的表格
\usepackage{caption} % 设置图片和表格的标题格式
\usepackage{subcaption} % 多个子图或表格
\usepackage{parskip}
\setlength{\parindent}{0pt}

\title{\textbf{光线投射实验报告}}
\author{张立博\ 2021012487}
\date{2023.4.10}

\begin{document}

\maketitle

\section{代码逻辑}
\begin{enumerate}
    \item 使用$SceneParser$解析场景,获取场景的相机、物体组和屏幕的宽度高度
    \item 循环屏幕中的像素,对于每个像素
        \begin{enumerate}
            \item 相机发射光线
            \item 计算光线与场景物体组的交点
            \item 如果相交,则遍历场景中所有光源使用Phong模型计算局部光强并累加,计算得到像素着色
            \item 如果不相交,则像素使用背景色
        \end{enumerate}
    \item 保存渲染生成的图片
\end{enumerate}

\section{代码参考}
未与同学进行讨论,未借鉴网上与其他同学的代码
\end{document}