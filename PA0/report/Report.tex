\documentclass[12pt,a4paper]{article}
\usepackage[margin=1in]{geometry} % 设置页面边距
\usepackage{ctex}
\usepackage{graphicx}
\usepackage{float}
\usepackage{amsmath}
\usepackage{amssymb} % 数学符号
\usepackage{booktabs} % 绘制漂亮的表格
\usepackage{array} % 数组和表格
\usepackage{multirow} % 多行合并的表格
\usepackage{caption} % 设置图片和表格的标题格式
\usepackage{subcaption} % 多个子图或表格
\usepackage{parskip}
\setlength{\parindent}{0pt}

\title{\textbf{光栅图形学作业}}
\author{张立博\ 2021012487}
\date{2023.3.21}

\begin{document}

\maketitle

\section{代码逻辑}

\subsection*{画线-Bresenham算法}
代码首先定义了起点坐标$xA$、$yA$和终点坐标$xB$、$yB$,并计算了x和y的增量方向$sx$和$sy$。然后使用while循环不断计算每个像素点的坐标,并将其绘制在图像上,直到达到终点坐标\\
在while循环内部,代码使用Bresenham算法中的增量计算方法来确定下一个要绘制的像素点的位置\\
具体来说,代码使用变量$err$和$e2$来表示当前像素点到直线的误差,$e2 = 2 * err$\\
然后根据直线的斜率关系来计算下一个要绘制的像素点的位置\\
如果$e2 > -dy$,则x的值加上sx;如果$e2 < dx$,则y的值加上sy,同时更新对应的$err$\\
这样可以避免小数与除法,并且可以绘制斜率不存在和为0的直线段,同时不用单独对特殊情况进行处理
\subsection*{画圆-扫描转换算法}
由于圆弧具有八对称性,所以只要扫描$1/8$圆弧就可以求出整个圆弧的像素集\\
方便起见,可以先考虑半径相同,以原点为圆心的圆,绘制时像素点进行平移\\
代码中以$(0,R)$为起点,顺时针为方向绘制八分圆\\
代码首先定义了圆心坐标$cx$、$cy$和半径$radius$,并初始化变量x、y和d。然后使用while循环不断计算圆上每个像素点的坐标,并将其绘制在图像上\\
在while循环内部,代码使用增量计算方法来确定下一个要绘制的像素点的位置\\
具体来说,代码使用变量d来表示当前像素点到圆形的误差,然后根据d和圆的形状关系来计算下一个要绘制的像素点的位置\\
代码将d初始化为$1-radius$预先计算两个增量值$deltaE$和$deltaSE$,如果d<0,则增量值为$deltaE$,否则增量值为$deltaSE$\\
这种方法可以使浮点数改为整数,将乘法运算改为加法运算,提高算法的效率
\subsection*{填充-非递归版本}
基于宽度优先搜索实现\\
代码首先初始化一个队列,将种子点入队;同时记录种子点像素的颜色为$oldColor$,若队列非空,则进入while循环\\
在while循环内部,获取队列首元素坐标为当前坐标并令其出队\\
若当前坐标像素颜色不等于$oldColor$,则跳过剩下的部分进行判断是否开始下一次循环\\
否则将当前坐标像素染为填充颜色,同时判断其四周点是否在绘图区域内以及颜色是否为$oldColor$,若是则令其入队\\
这种基于宽度优先搜索的四联通填充算法可以避免递归实现,提高了效率
\section{代码参考}
未与同学进行讨论与网上借鉴\\
\section{问题}
1.像素点位置确定有误:光栅图形学中像素繁多,需要在代码中精确定位每一个像素坐标才能完成较为标准的图形\\
2.通过误差确定下一个绘制的像素点:需要较多通分和运算确定比较的关系式,同时要尽量用整数和乘除替换浮点数和加减\\
\end{document}